\documentclass[a4paper,12pt]{article}
\usepackage[utf8]{inputenc}
\usepackage[brazil]{babel}
\usepackage[T1]{fontenc}
\usepackage{geometry}
\usepackage{hyperref}
\usepackage{listings}
\usepackage{xcolor}

\geometry{margin=1in}

\lstset{
    backgroundcolor=\color{gray!10},
    basicstyle=\ttfamily\small,
    frame=single,
    breaklines=true,
    columns=fullflexible,
    upquote=true
}

\title{\Huge Sistema de Recomendação de Produtos}
\author{Universidade de Brasília -- DCC \\ Projeto de IA 2025/1}
\date{}

\begin{document}
\maketitle

\section*{Visão Geral}
Este projeto conecta pequenos produtores a consumidores de uma região via
recomendação colaborativa KNN, filtragem geográfica e preferências.

\section*{Funcionalidades}
\begin{itemize}
    \item Recomendações baseadas no histórico de clientes similares.
    \item Interface web para seleção de clientes e exibição de recomendações.
    \item Visualização do histórico de compras.
    \item API REST para integração.
\end{itemize}

\section*{Pré-requisitos}
\begin{itemize}
    \item Docker instalado.
    \item Arquivo \texttt{sells\_data.csv} em \texttt{data/}.
\end{itemize}

\section*{Docker}
\begin{lstlisting}[language=bash]
# Build da imagem
docker build -t recommendation-api .

# Executar container
docker run -p 5000:5000 \
  -v $(pwd)/data:/app/data \
  recommendation-api
\end{lstlisting}
Obs.: se a porta 5000 estiver ocupada, use \texttt{-p 5001:5000}.

\section*{Como Executar}

\subsection*{Serviço principal (Flask + KNN)}
\begin{lstlisting}[language=bash]
python ai.py [-f path/to/best_params.json]
python app.py
\end{lstlisting}
Acesse em: \url{http://localhost:5000}

\subsection*{Testbench}
\begin{itemize}
  \item Modo finito:
    \begin{lstlisting}[language=bash]
python testbench.py -n 50
    \end{lstlisting}
    Gera \texttt{testbench\_50/logs/Exec\_\#\#/} e \texttt{testbench\_50/metrics.json}.
  \item Modo indefinido:
    \begin{lstlisting}[language=bash]
python testbench.py -i
    \end{lstlisting}
    Atualiza continuamente o diretório \texttt{best/} até interrupção (\texttt{Ctrl+C}).
\end{itemize}

\subsection*{Grid Search}
\begin{lstlisting}[language=bash]
python grid_search.py --out grid_results --reps 5
\end{lstlisting}
Gera:
\begin{itemize}
  \item \texttt{grid\_results/logs/combo\_/run\_/}
  \item \texttt{grid\_results/best\_params.json}
  \item \texttt{grid\_results/grid\_metrics.json}
\end{itemize}

\section*{Parâmetros e Flags}
\begin{tabular}{p{4cm} p{4cm} p{6cm}}
\hline
\textbf{Script} & \textbf{Flag} & \textbf{Descrição} \\
\hline
\texttt{ai.py} & \texttt{-f, --best-params} & Carrega hiperparâmetros de \texttt{best\_params.json}. \\
\texttt{testbench.py} & \texttt{-n, --n\_runs} & Número de iterações finitas (padrão: 100). \\
 & \texttt{-i, --indefinite} & Modo indefinido até \texttt{Ctrl+C}. \\
\texttt{grid\_search.py} & \texttt{-o, --out} & Diretório base para logs/resultados (padrão: \texttt{grid\_search/}). \\
 & \texttt{-r, --reps} & Número de runs por combinação (padrão: 10). \\
\hline
\end{tabular}

\section*{Testes \& Validação}
\begin{itemize}
  \item \textbf{Precision@K}: fração de itens recomendados que são relevantes.
  \item \textbf{Recall@K}: fração de itens relevantes que foram recomendados.
  \item Use \texttt{testbench.py} para medir estabilidade (ex.: média de retries).
  \item Use \texttt{grid\_search.py} para otimizar hiperparâmetros (\texttt{K\_VIZINHOS}, \texttt{K\_RECS}, \texttt{ALPHA}, etc.).
\end{itemize}

\section*{Métricas de Performance}
O arquivo \texttt{grid\_metrics.json} contém:
\begin{itemize}
  \item \textbf{best}: melhores parâmetros e métricas.
  \item \textbf{combos}: resumo de cada combinação.
  \item \textbf{time\_metrics}: 
    \begin{itemize}
      \item \texttt{winner\_run\_time}: tempo do run vencedor.
      \item \texttt{winner\_combo\_total\_time}: tempo total do combo vencedor.
      \item \texttt{winner\_combo\_pct\_of\_grid}: percentual do grid.
      \item \texttt{combo\_times}: tempo total e percentual de cada combo.
      \item \texttt{total\_grid\_time}: duração completa do grid.
    \end{itemize}
\end{itemize}

\section*{Licença}
Liberado sob uma licença de uso geral no contexto da UnB. \newline Consulte \texttt{LICENSE} para obter mais detalhes.

\end{document}
